\section{Linux Philosophy}

\begin{frame}
   {Linux Philosophy}

   \begin{itemize}
      \item \textbf{Linux} is \textit{not} \textbf{UNIX}, although
      its creators were rooted in it.
      \item Technically, \textbf{Linux} is only the \textbf{kernel}.  Complete
      operating system includes:
      \begin{itemize}
         \item Major Libraries
         \item Important Applications and Utilities
         \item User Interface
         \item etc.
      \end{itemize}
      \item Follows major \textbf{UNIX} practices:
      \begin{itemize}
         \item ``Everything is a file''
         \item Small tools that do one thing well
         \item Tools can be linked together via pipes
      \end{itemize}
      \item Collaborative , Open Source, development
   \end{itemize}

\end{frame}

\cprotect\note{

   \textbf{Linux} borrows heavily from the \textbf{UNIX}
   operating system.

   \textbf{Linux} is a fully multitasking, multi-user
   operating system, with built-in networking and service
   processes (known as daemons in the \textbf{UNIX} world).

   Files are stored in a hierarchical file system, with the
   top node of the system being \textbf{root} or simply
   \texttt{/}.

   Whenever possible, \textbf{Linux} makes its components
   through a file-like interface.  Processes, devices and
   network sockets are look like files, and can often be worked
   with using the same utilities used for regular
   files.

   \textbf{Linux} is developed by a loose confederation of
   developers from all over the world, collaborating over
   the Internet, with Linus Torvalds at the head.
   Technical skill and a desire to contribute are the only
   qualifications for participating.
}



