\clearpage\section{Labs}\begin{Lab}

\begin{exe} {\textbf{Linux} philosophy}

   According to \textbf{UNIX} design philosophy, which is
   used by \textbf{Linux}:

   ``\textit{everything is a \underline{\hspace*{4em}}}''?

   \begin{sol}

      \textbf{Linux} has always adopted the file paradigm to
      deal with all sorts of objects, such as when doing I/O
      to devices or network sockets and pipes.  It also uses
      many so-called pseudo-filesystems to implement core
      functionalities.

      At the same time, \textbf{Linux} systems also follow
      the traditional \textbf{UNIX} philosophy of using many
      relatively simple tools that do one thing well, and
      then connecting these tools through pipes.

   \end{sol}
\end{exe}

\begin{exe} {Operating System Components}

   Other than the kernel, name three components that can be
   found in a \textbf{Linux} distribution.

   \begin{sol}

      Most \textbf{Linux} distributions will contain a
      \textbf{Linux} kernel along with:
      \begin{enumerate}
         \item
         Tools for file-related operations, and user and
         software package management.
         \item
         A graphical environment (GUI).
         \item
         Optional software packages.
      \end{enumerate}
   \end{sol}
\end{exe}

\begin{exe} {\textbf{Debian}-based \textbf{Linux} distributions}

   Name another \textbf{Debian}-based distribution besides
   \textbf{Ubuntu}.

   \begin{sol}

      \textbf{Linux Mint}
   \end{sol}
\end{exe}

\begin{exe} {\textbf{Red Hat}-based \textbf{Linux} distributions}

   Name one or more free distributions that are derived
   entirely from the \textbf{Red Hat Enterprise Linux}
   distribution.

   \begin{sol}

      \textbf{CentOS}, \textbf{Scientific Linux}.

   \end{sol}
\end{exe}

\begin{exe} {How do commercial \textbf{Linux} distributions Make Money?}

   \textbf{Red Hat} was the first \textbf{Linux} based
   company to surpass 1 billion dollars in annual revenue.
   If \textbf{Linux} is freely available, how do companies
   like \textbf{Red Hat} make money?

   \begin{sol}

      \textbf{Red Hat} and other commercial \textbf{Linux}
      distributions make money by selling services such as
      support, consulting, updates, and hardware and software
      certification.

   \end{sol}
\end{exe}

\begin{exe} {Open Source Contribution}

   What training or certification is required before
   someone is able to contribute to an Open Source project?

   \begin{sol}

      \textbf{None}.  Anyone may contribute who doesn't mind sharing
      their abilities with the community.  The only hurdle:
      convincing the community that their contributions are
      useful.

   \end{sol}
\end{exe}

\begin{exe} {Open Source Projects}

   Name some other widely used open source software
   projects, other than \textbf{Linux}.

   \begin{sol}

      \textbf{Apache}, \textbf{MySQL}, \textbf{PHP},
      \textbf{LibreOffice}, \textbf{GCC}, \textbf{Python}
      etc.

      The list is endless!

   \end{sol}
\end{exe}

\end{Lab}

