\section{The Command Line}

\begin{frame}
   {Needed Tools}

   \begin{itemize}
      \item
      We need to use some tools before we learn about them!
      \item
      Otherwise it will be difficult to accomplish even simple
      tasks
      \item
      We will introduce some subjects briefly here, without
      much explanation
      \item
      We will explain in detail later
   \end{itemize}

\end{frame}

\cprotect\note{

   Trying to learn \textbf{axiomatically} (never doing any
   forward referencing, making sure everything mentioned
   has already been taught) is both boring and inefficient.

   All students have probably had significant previous
   experience either with other operating systems, or with
   \textbf{Linux} itself.

   Thus we are going to orient ourselves with the use of
   certain tools right away, so we can begin doing useful
   work from the start.  Later we will discuss each of
   these items in more detail.

}



\begin{frame}
   {Command Line and Terminal Windows}
   \begin{itemize}
      \item
      To issue one particular command:
      \begin{itemize}
         \item
         Hit \verb?Alt-F2?
         \item
         Type in the command and hit \verb?Return?
      \end{itemize}
      \item
      To open up a \textbf{Terminal Window} for issuing
      commands and running programs at the \textbf{command
      line}:
      \begin{itemize}
         \item
         Find \verb?Terminal? on your desktop's graphical menu
         system and click on it; or
         \item
         Using the \verb?Alt-F2? method describe above, type
         \textbf{gnome-terminal} for \textbf{GNOME} desktops,
         or \textbf{konsole} for \textbf{KDE} desktops
         \item
         Right click on the desktop background and click on
         \verb?Open Terminal?. (On most desktops)
      \end{itemize}
      \item
      Try to open a terminal.

   \end{itemize}
\end{frame}

\cprotect\note{

   We are going to discuss the graphical desktop interface
   before going deeply into the command line text
   interface, but there are some things which are just
   either very difficult or tedious to do through a
   \textbf{GUI} (\textbf{G}raphical \textbf{U}ser
   \textbf{I}nterface), or downright impossible.

   However, many of these tasks are quite easy to do at a
   command line. \textbf{Linux} system administrators use
   command line utilities for system configuration and
   control far more than they do graphical ones, so we
   might as well get used to deploying the power of the
   command line.



}


\begin{frame}
   {Editors}
   \begin{itemize}
      \item
      From the outset there will be times we need to create or
      edit text files
      \item
      Later we will have a whole section on editors, including
      the two most widely used, \textbf{vi} and \textbf{emacs}
      \item
      If you are already familiar with one of these find.  Otherwise,
      for now it is easy to use:
      \begin{itemize}
         \item
         \textbf{nano}: a command line terminal editor, no
         learning curve, very easy to use
         \item
         \textbf{gedit}: a graphical editor similar to
         \textbf{Notepad}
      \end{itemize}      \
      \item You can launch either from the command line as
      mentioned earlier, or find in the menu system, usually
      under \verb?Accessories?
   \end{itemize}

\end{frame}

\cprotect\note{

   \textbf{vi} has been on all \textbf{UNIX}-related systems
   since their inception and can be rather compact, although
   the versions in place on major distributions are actually
   very robust

   \textbf{emacs} has also been around since the earliest days
   and usually is available if not installed by default.

   These editors are the same on all operating systems and
   distributions and all experienced users use one or the
   other, and sometimes fight about which is better.

   \textbf{nano} and \textbf{gedit} are quite capable and easy
   to use as well.

   Other text editors are also available as we discuss later.
}

